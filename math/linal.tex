\newtheorem{definition}{Definition}

\newcommand{\RR}{\mathbb{R}}
\renewcommand{\>}{\rangle}
\newcommand{\<}{\langle}

\newcommand{\ket}[1]{|#1\rangle}
\newcommand{\bra}[1]{\langle #1 |}

\renewcommand{\vec}[1]{\boldsymbol{\mathbf{#1}}} % This produces upright bold letters, and works even for greek symbols
\newcommand{\grad}{\vec \nabla}

% Making notes
\newcommand{\todo}[1]{{\color{blue}[Todo] #1}}
\newcommand{\tocheck}[1]{{\color{red} #1}}
\newcommand{\comment}[1]{{\color{cyan} #1}}

\begin{document}
\chapter{Vector Spaces}

A subset of $W$ of $\RR^n$ is a \emph{subspace} if it has the following proerties:
\begin{enumerate}[(a)]
    \item If $w,w'\in W$, then $w+w'\in W$
    \item If $w\in W,\, c\in \RR$, then $cw\in W$
    \item The zero vector is in $W$
\end{enumerate}
Note that (c) seems to be a special case of (b), by putting $c=0$. However, it is (c) which ensures that $W$ is not the empty set.

\section{Fields}
\label{sec:fields}

The quintessential model for a field is the set of all complex numbers $\CC$.

\begin{definition}
    A \emph{field} $F$ is a set together with two composition laws
    \begin{equation}
       F\times F \xrightarrow{+} F \quad\text{and}\quad F\times F \xrightarrow{\times} F
    \end{equation}
    called \emph{addition}: $a+b \mapsto a+b$ and \emph{multiplication}: $a\times b \mapsto ab$, which satisfy these axioms
    \begin{enumerate}[(i)]
        \makethislistcompact % defined in the preamble

        \item $F$ with addition, written as $F^+$, is an abelian group,
        \item $F/\{0\}$ with multiplication, written as $F^\times$, is an abelian group,
        \item \emph{distributive law: } For all $a,b,c\in F$, we have $a(b+c) = ab + ac$.
    \end{enumerate}
\end{definition}
Note that axiom (iii) relates multiplication and addition.


A very interesting example of a field is 
\begin{align*}
    \FF_p &= \{\overline 0, \overline 1 ,\dotsc, \overline {p-1}\} = \ZZ/\ZZ_p,
\end{align*}
where $p$ is a prime number. 
\begin{lemma}
   The characteritic of any field F is either zero or a prime number. 
\end{lemma}
\begin{proof}
   Assume that the characteristic $m$ is neither zero nor prime. Then, it can be written as $m=rs$ for some positive integers $r,s$. Since we have
   \begin{align*}
        0 &= \overbrace{1 + \cdots + 1}^{m \text{ times}} 
        = \overbrace{1 + \cdots + 1}^{r \text{ times}}  + \cdots  +1 .
   \end{align*}
   Writing $\overbrace{1 + \cdots + 1}^{r \text{ times}}=a$, we get 
   \begin{equation*}
        0 = \overbrace{a + \cdots + a}^{s \text{ times}} 
        = a (\overbrace{1 + \cdots + 1}^{s \text{ times}}) \\
   \end{equation*}
   Now, either 
   \begin{align*}
        \overbrace{1 + \cdots + 1}^{s \text{ times}} = 0 \quad \text{ or }\quad
        a = \overbrace{1 + \cdots + 1}^{r \text{ times}} =0.
   \end{align*}
   In either case, we have a contradiction.
\end{proof}

% section fields (end)



\section{Problems}
\label{sec:problems}
\textbf{1.8}\\
Let $p$ be a prime integer.
\begin{enumerate}[(a)]
    \item Fermat's theorem: $a^p \equiv a \mod p$ for every integer $a$.\\
        This is a direct consequence of the fact that $\FF_p^\times$ is a cyclic group of order $p-1$.
    \item Wilson's theorem: $(p-1)! \equiv -1 \mod p$.\\
        The case with $p=2$ is trivial. Let $p>2$, which is odd.
        Let $a$ be a primitive root of $\FF_p^\times$. We have
        \begin{align*}
            \{1,\dotsc,p-1\} &= \{a^1,\dotsc,a^{p-1}\} \\
            \implies (p-1)! &= a^1 \cdots a^{p-2} a^{p-1}\\
                &= a^{\frac{p(p-1)}{2}} 
        \end{align*}
        Since, $a^{p}\equiv a \mod p$ (by Fermat's theorem), then $(a^{p})^{(p-1)/2}\equiv a^{(p-1)/2} \mod p$. 

        For some integer $x$, if we have $x^2 \equiv 1\mod p$, then
        \begin{align*}
            x^2 - 1 = (x-1)(x+1)  \equiv 0 \mod p\\
            \implies x \equiv 1 \mod p\quad\text{or}\quad x\equiv -1 \mod p
        \end{align*}
        If $x=a^{(p-1)/2}$, then $x=1$ would mean that $a$ is not a primitive root, which is a contradiction. So, $a^{(p-1)/2} \equiv -1 \mod p$. Thus,
        \begin{align*}
            (p-1)! = a^{\frac{p(p-1)}{2}} &\equiv a^{\frac{p-1}{2}} \mod p \\
            &\equiv -1 \mod p
        \end{align*}
        \qedsymbol
\end{enumerate}


% section problems (end)


\end{document}
